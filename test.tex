% Options for packages loaded elsewhere
\PassOptionsToPackage{unicode}{hyperref}
\PassOptionsToPackage{hyphens}{url}
%
\documentclass[
]{article}
\usepackage{amsmath,amssymb}
\usepackage{iftex}
\ifPDFTeX
  \usepackage[T1]{fontenc}
  \usepackage[utf8]{inputenc}
  \usepackage{textcomp} % provide euro and other symbols
\else % if luatex or xetex
  \usepackage{unicode-math} % this also loads fontspec
  \defaultfontfeatures{Scale=MatchLowercase}
  \defaultfontfeatures[\rmfamily]{Ligatures=TeX,Scale=1}
\fi
\usepackage{lmodern}
\ifPDFTeX\else
  % xetex/luatex font selection
\fi
% Use upquote if available, for straight quotes in verbatim environments
\IfFileExists{upquote.sty}{\usepackage{upquote}}{}
\IfFileExists{microtype.sty}{% use microtype if available
  \usepackage[]{microtype}
  \UseMicrotypeSet[protrusion]{basicmath} % disable protrusion for tt fonts
}{}
\makeatletter
\@ifundefined{KOMAClassName}{% if non-KOMA class
  \IfFileExists{parskip.sty}{%
    \usepackage{parskip}
  }{% else
    \setlength{\parindent}{0pt}
    \setlength{\parskip}{6pt plus 2pt minus 1pt}}
}{% if KOMA class
  \KOMAoptions{parskip=half}}
\makeatother
\usepackage{xcolor}
\usepackage{color}
\usepackage{fancyvrb}
\newcommand{\VerbBar}{|}
\newcommand{\VERB}{\Verb[commandchars=\\\{\}]}
\DefineVerbatimEnvironment{Highlighting}{Verbatim}{commandchars=\\\{\}}
% Add ',fontsize=\small' for more characters per line
\newenvironment{Shaded}{}{}
\newcommand{\AlertTok}[1]{\textcolor[rgb]{1.00,0.00,0.00}{\textbf{#1}}}
\newcommand{\AnnotationTok}[1]{\textcolor[rgb]{0.38,0.63,0.69}{\textbf{\textit{#1}}}}
\newcommand{\AttributeTok}[1]{\textcolor[rgb]{0.49,0.56,0.16}{#1}}
\newcommand{\BaseNTok}[1]{\textcolor[rgb]{0.25,0.63,0.44}{#1}}
\newcommand{\BuiltInTok}[1]{\textcolor[rgb]{0.00,0.50,0.00}{#1}}
\newcommand{\CharTok}[1]{\textcolor[rgb]{0.25,0.44,0.63}{#1}}
\newcommand{\CommentTok}[1]{\textcolor[rgb]{0.38,0.63,0.69}{\textit{#1}}}
\newcommand{\CommentVarTok}[1]{\textcolor[rgb]{0.38,0.63,0.69}{\textbf{\textit{#1}}}}
\newcommand{\ConstantTok}[1]{\textcolor[rgb]{0.53,0.00,0.00}{#1}}
\newcommand{\ControlFlowTok}[1]{\textcolor[rgb]{0.00,0.44,0.13}{\textbf{#1}}}
\newcommand{\DataTypeTok}[1]{\textcolor[rgb]{0.56,0.13,0.00}{#1}}
\newcommand{\DecValTok}[1]{\textcolor[rgb]{0.25,0.63,0.44}{#1}}
\newcommand{\DocumentationTok}[1]{\textcolor[rgb]{0.73,0.13,0.13}{\textit{#1}}}
\newcommand{\ErrorTok}[1]{\textcolor[rgb]{1.00,0.00,0.00}{\textbf{#1}}}
\newcommand{\ExtensionTok}[1]{#1}
\newcommand{\FloatTok}[1]{\textcolor[rgb]{0.25,0.63,0.44}{#1}}
\newcommand{\FunctionTok}[1]{\textcolor[rgb]{0.02,0.16,0.49}{#1}}
\newcommand{\ImportTok}[1]{\textcolor[rgb]{0.00,0.50,0.00}{\textbf{#1}}}
\newcommand{\InformationTok}[1]{\textcolor[rgb]{0.38,0.63,0.69}{\textbf{\textit{#1}}}}
\newcommand{\KeywordTok}[1]{\textcolor[rgb]{0.00,0.44,0.13}{\textbf{#1}}}
\newcommand{\NormalTok}[1]{#1}
\newcommand{\OperatorTok}[1]{\textcolor[rgb]{0.40,0.40,0.40}{#1}}
\newcommand{\OtherTok}[1]{\textcolor[rgb]{0.00,0.44,0.13}{#1}}
\newcommand{\PreprocessorTok}[1]{\textcolor[rgb]{0.74,0.48,0.00}{#1}}
\newcommand{\RegionMarkerTok}[1]{#1}
\newcommand{\SpecialCharTok}[1]{\textcolor[rgb]{0.25,0.44,0.63}{#1}}
\newcommand{\SpecialStringTok}[1]{\textcolor[rgb]{0.73,0.40,0.53}{#1}}
\newcommand{\StringTok}[1]{\textcolor[rgb]{0.25,0.44,0.63}{#1}}
\newcommand{\VariableTok}[1]{\textcolor[rgb]{0.10,0.09,0.49}{#1}}
\newcommand{\VerbatimStringTok}[1]{\textcolor[rgb]{0.25,0.44,0.63}{#1}}
\newcommand{\WarningTok}[1]{\textcolor[rgb]{0.38,0.63,0.69}{\textbf{\textit{#1}}}}
\setlength{\emergencystretch}{3em} % prevent overfull lines
\providecommand{\tightlist}{%
  \setlength{\itemsep}{0pt}\setlength{\parskip}{0pt}}
\setcounter{secnumdepth}{5}
\newlength{\cslhangindent}
\setlength{\cslhangindent}{1.5em}
\newlength{\csllabelwidth}
\setlength{\csllabelwidth}{3em}
\newlength{\cslentryspacingunit} % times entry-spacing
\setlength{\cslentryspacingunit}{\parskip}
\newenvironment{CSLReferences}[2] % #1 hanging-ident, #2 entry spacing
 {% don't indent paragraphs
  \setlength{\parindent}{0pt}
  % turn on hanging indent if param 1 is 1
  \ifodd #1
  \let\oldpar\par
  \def\par{\hangindent=\cslhangindent\oldpar}
  \fi
  % set entry spacing
  \setlength{\parskip}{#2\cslentryspacingunit}
 }%
 {}
\usepackage{calc}
\newcommand{\CSLBlock}[1]{#1\hfill\break}
\newcommand{\CSLLeftMargin}[1]{\parbox[t]{\csllabelwidth}{#1}}
\newcommand{\CSLRightInline}[1]{\parbox[t]{\linewidth - \csllabelwidth}{#1}\break}
\newcommand{\CSLIndent}[1]{\hspace{\cslhangindent}#1}
\ifLuaTeX
  \usepackage{selnolig}  % disable illegal ligatures
\fi
\IfFileExists{bookmark.sty}{\usepackage{bookmark}}{\usepackage{hyperref}}
\IfFileExists{xurl.sty}{\usepackage{xurl}}{} % add URL line breaks if available
\urlstyle{same}
\hypersetup{
  pdftitle={Approfondissement -- LaTeX et outils connexes},
  pdfauthor={Sébastien Mengin},
  hidelinks,
  pdfcreator={LaTeX via pandoc}}

\title{Approfondissement -- LaTeX et outils connexes}
\author{Sébastien Mengin}
\date{}

\begin{document}
\maketitle

\hypertarget{questions-techniques-du-jour}{%
\section{Questions techniques du
jour}\label{questions-techniques-du-jour}}

\hypertarget{gestion-des-documents-volumineux}{%
\subsection{Gestion des documents
volumineux}\label{gestion-des-documents-volumineux}}

\begin{itemize}
\tightlist
\item
  commandes \texttt{\textbackslash{}input} et
  \texttt{\textbackslash{}include}
\item
  commande \texttt{\textbackslash{}includeonly}
\item
  dossiers pour les éléments graphiques et commande
  \texttt{\textbackslash{}graphicspath\{\}}
\item
  package \texttt{comment}
\end{itemize}

\hypertarget{conversions-en-latex-et-word}{%
\subsection{Conversions en LaTeX et
Word}\label{conversions-en-latex-et-word}}

\begin{itemize}
\tightlist
\item
  pratiques actuelles
\item
  améliorations possibles ? (scripts ?)
\end{itemize}

Avec la ligne suivante les sections ne sont pas numérotées :

\begin{verbatim}
pandoc --bibliography=Biblio_These.bib --bibliography=ComplementBibTex -o RevueLit.docx main.tex --citeproc --number-sections
\end{verbatim}

\begin{itemize}
\tightlist
\item
  utiliser LaTeX pour la mise en page finale\footnote{Par conséquent,
    rédiger en markdown le reste du temps\ldots{}}
\end{itemize}

\hypertarget{bibliographies}{%
\subsection{Bibliographies}\label{bibliographies}}

Convertir un DOI en bibtex. \url{https://www.doi2bib.org/}

\hypertarget{saisie-en-markdown}{%
\subsection{Saisie en Markdown}\label{saisie-en-markdown}}

\begin{itemize}
\item
  un exemple :
  \url{https://opensource.com/article/18/9/pandoc-research-paper}
\item
  gestion des bibliographies avec Zotero et l'extension better-bibtex
  (Smith and Doe 2023)
\item
  exemple de commande à saisir dans un terminal pour exporter en LaTeX :

\begin{Shaded}
\begin{Highlighting}[]
\NormalTok{pandoc {-}f markdown {-}t latex {-}{-}biblatex {-}s FO{-}INRAE{-}approfondissement.md {-}o FO{-}INRAE{-}approfondissement.tex}
\end{Highlighting}
\end{Shaded}
\item
  éditeur collaboratif en temps réel markdown

  \begin{itemize}
  \tightlist
  \item
    stylo (\url{https://stylo.huma-num.fr/})
  \end{itemize}
\end{itemize}

\hypertarget{ruxe9daction-collaborative-en-temps-ruxe9el-avec-overleaf}{%
\section{Rédaction collaborative en temps réel avec
Overleaf}\label{ruxe9daction-collaborative-en-temps-ruxe9el-avec-overleaf}}

\begin{itemize}
\tightlist
\item
  limites d'Overleaf en version gratuite : 1 collaborateur max avec
  droit d'écriture\footnote{Une note de bas de page peut être balisée
    avec un chiffre ou avec un mot, \texttt{pandoc} se charge des
    compteurs\ldots{}}
\end{itemize}

\begin{itemize}
\tightlist
\item
  solutions alternatives :

  \begin{itemize}
  \tightlist
  \item
    gérer les droits d'accès aux différents collaborateurs (écriture,
    lecture)
  \item
    utiliser des gestionnaires de version (github et cie). Limite :
    asynchrone
  \end{itemize}
\end{itemize}

\hypertarget{ruxe9capitulatif-benchmark}{%
\section{Récapitulatif (benchmark)}\label{ruxe9capitulatif-benchmark}}

\hypertarget{nextcloud-inrae}{%
\subsection{Nextcloud INRAE}\label{nextcloud-inrae}}

\begin{itemize}
\tightlist
\item
  markdown ok
\item
  zotero ? fichiers de bibliographie ?
\item
  images ?
\item
  historique des modifications ?
\item
  validation des modifications ?
\end{itemize}

\hypertarget{github}{%
\subsection{Github ?}\label{github}}

\hypertarget{overleaf-version-payante}{%
\subsection{Overleaf version payante}\label{overleaf-version-payante}}

\hypertarget{ruxe9daction-des-thuxe8ses}{%
\subsubsection{Rédaction des thèses}\label{ruxe9daction-des-thuxe8ses}}

Overleaf pour le côté pragmatique, en version payante. Voir pour une
version hébergée sur les serveurs INRAE ?

\hypertarget{ruxe9daction-darticles-de-recherches}{%
\subsubsection{Rédaction d'articles de
recherches}\label{ruxe9daction-darticles-de-recherches}}

Développer l'instance markdown sur nextcloud INRAE ?

\hypertarget{perspectives}{%
\section{Perspectives}\label{perspectives}}

Selon nouvelle session de formation 2025-2026 à soumettre.

\hypertarget{utilisation-des-templates-latex}{%
\subsection{Utilisation des templates
LaTeX}\label{utilisation-des-templates-latex}}

\begin{itemize}
\tightlist
\item
  rédiger en format basique (article, report, etc.) pour la
  compatibilité maximale avec d'autres templates
\item
  appliquer un template déterminé lorsqu'il existe ou est fournir par
  une revue
\item
  créer des templates propres aux articles et travaux publiés pour
  Dynafor :

  \begin{itemize}
  \tightlist
  \item
    template écologie
  \item
    template SHS
  \item
    interdisciplinaire
  \end{itemize}
\end{itemize}

\hypertarget{groupe-de-travail-latex}{%
\subsection{Groupe de travail LaTeX}\label{groupe-de-travail-latex}}

Piloté par infogéom, animé par les utilisateurs LaTeX du laboratoire.

\hypertarget{calendrier-prochaines-sessions}{%
\section{Calendrier prochaines
sessions}\label{calendrier-prochaines-sessions}}

\hypertarget{juin}{%
\subsection{Juin}\label{juin}}

13, 16, 20, 24, 25

\hypertarget{juillet}{%
\subsection{Juillet}\label{juillet}}

9, 10, 18, 21, 24, 25

\hypertarget{refs}{}
\begin{CSLReferences}{1}{0}
\leavevmode\vadjust pre{\hypertarget{ref-smith2023economic}{}}%
Smith, John, and Jane Doe. 2023. {``Economic Statistics and Their Impact
on Policy Making.''} \emph{Journal of Economic Research} 45 (2):
123--45.

\end{CSLReferences}

\end{document}
